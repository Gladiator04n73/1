%https://ru.overleaf.com/learn

% Логика документа

% Выбираем класс документа и классовые опции 
% Класс определяет вид и структуру документа. 
% Класс -- это база, которую можно править с помощью подключаемых стилевых файлов. 
% В классе задаётся геометрия страницы и определены команды секционирования. 

% Имеются следующие базовые классы:
% article (статья), 
% book (книга), 
% report (отчёт), 
% proc (доклад), 
% letter (письмо) и 
% slides (слайды)!.

% Стандартные стилевые опции:
% 10pt|11pt|12pt — установка базового размера шрифта. 
% a4paper — установка размера листа бумаги. Следует использовать всегда, так как по умолчанию LATEX использует размер листа letter.
% draft — режим черновой печати для «отлавливания» проблем вёрстки. 
% oneside|twoside — форматирование документа для односторонней и двухсторонней печати, соответственно.
% twocolumn — печать в две колонки.

% Модификации классов:
% extarticle, extbook, extletter, extproc, extreport (8pt, 9pt, 14pt, 17pt и 20pt)

\documentclass[a4paper, 12pt]{article} 
% Пакеты подключаются командой 
% \usepackage[необязательные параметры]{имя пакета}

% Выбор кодировки файла
\usepackage[utf8]{inputenc}

\usepackage{amsmath, amssymb}

%Выбор кодировки ширифтов
%для тестов только на русском языке этот параметр можно не указывать
\usepackage[T2A]{fontenc} 
% Подключение языков документа
% Последний язык доминирующий
\usepackage[english, russian]{babel}  

%Задаем параметры документа
\usepackage[top = 20 mm, 
            bottom = 20 mm, 
            left = 30 mm, 
            right = 30 mm]{geometry}
            
%Красная строка в первом абзаце
\usepackage{indentfirst}

%Величина отступа красной строки
\setlength{\parindent}{12.5 mm}

%Межстрочный интервал
%\def\baselinestretch{1.5}
\usepackage{setspace}
\setstretch{1}

\setcounter{equation}{61}
 
\begin{document}
Слагаемое $\alpha u_t$ в правой части уравнения соответствует трению, пропорциональному скорости. 

Рассмотрим сначала задачу о распространении периодического граничного режима:

\begin{equation} \label{f62}
u(l,\,t)= A\,\cos\, \omega t\, (\, \text{или}\, u(l,\,t)    = B\,\cos\, \omega t\,), 
\end{equation}
\begin{equation} \label{f63}
(0,\,t)= 0.    \setcounter{equation}{63} 
\end{equation}

Для дальнейшего нам удобнее записать граничное условие в комплексной форме

\begin{equation} \label{f64}(l,\,t)= Ae^{i\omega t}.
\end{equation}

Если \begin{displaymath} (x,\,t)= u^{(1)}(x,\,t)+iu^{(2)}(x,\,t).
\end{displaymath}
удовлетворяет уравнению  (61) с граничными условиями \eqref{f63} и \eqref{f64}, то \, \(u^{(1)} (x,\,t)\)\, \text{и}  \(u^{(2)} (x,\,t)\)\,  - его действительная и мнимая части удовлетворяют тому же уравнению (в силу его линейности), условию \eqref{f63} и граничными условиями при х = l

\begin{displaymath} u^{(1)}\,(l,\,t) =A\,\cos\, \omega t\,,\end{displaymath}
\begin{displaymath}
u^{(2)}\,(l,\,t) =A\,\sin\, \omega t\,.
\end{displaymath}
Итак, найдем решение задачи

\begin{equation}
\left.
\begin{array}{ll}
 u_{tt} &= \alpha^2u_{xx}-\alpha^u_{t},\\
 u(0,\,t) &=0,\\
 u(l,\,t) &= Ae^{i\omega t}.
\end{array}
\right\}
\end{equation}
Полагая
\begin{displaymath} u(x,\,t)= X\,(x)\, e^{i\omega t}\,\end{displaymath}
 подставляя это выражение в уравнение, получим для функции X (x) следующую задачу:
 \begin{equation} \label{f66}X''+k^{2}X=0.\,\,\,(k^{2}=\left(\frac{\omega^2}{\alpha^3}-ia\frac{\omega}{\alpha^3}\right),
\end{equation}
 \begin{equation} \label{f67}X\,(0)=0,
\end{equation}
 \begin{equation} \label{f68}X\,(l)=A.
\end{equation}
Из уравнения \eqref{f66} и граничного условия \eqref{f67} находим:
 \begin{displaymath} X\,(x)=C\,\sin\,kx.
\end{displaymath}
Условие при x=l дает: \begin{equation} \label{f69}C=\frac{A}{\sin\,kl},
\end{equation}
так что\begin{equation} \label{f70} X\,(x)=A\frac{\sin\,kx}{\sin\,kl}=X_{1}(x)+iX_{2}(x),
\end{equation}
где \(X_{1}(x)\) \, и  \(X_{2}(x)\) \,  - действительная и мнимая части X (x).

Искомое решение можно представить в виде  \begin{displaymath}
u(x,\,t)=[X_{1}(x)\,+\,iX_{2}(x)]e^{i\omega t}=u^{(1)}(x,\,t)+iu^{(2)}(x,\,t),
\end{displaymath}
где \begin{displaymath}
u^{(1)}(x,\,t)=X_{1}(x)\cos\omega t\,-\,X_{2}(x)\sin\omega t]+iu^{(2)}(x,\,t).,
\end{displaymath}
\begin{displaymath}
u^{(2)}(x,\,t)=X_{1}(x)\sin\omega t\,+\,X_{2}(x)\cos\omega t]+iu^{(2)}(x,\,t).
\end{displaymath}
Переходя у пределу при а \( а \to 0\) \,, \text{найдем, что}

\begin{equation} \label{f70} \bar{k}\,= \lim\limits_{a \to 0} k =\frac{\omega}{\alpha}
\end{equation}
и, соответственно,\begin{equation} \label{f70} \bar{u}^{(1)}\,(x,\,t)= \lim\limits_{a \to 0} u^{(1)}\,(x,\,t) =A\frac{\sin\frac{\omega}{\alpha}x} {\sin\frac{\omega}{\alpha}l}\cos\omega t,
\end{equation}\begin{equation} \label{f70} \bar{u}^{(2)}\,(x,\,t)= \lim\limits_{a \to 0} u^{(2)}\,(x,\,t) =A\frac{\sin\frac{\omega}{\alpha}x} {\sin\frac{\omega}{\alpha}}\sin\omega t.
\end{equation}
Рассмотрим следующую задачу:\begin{equation*}
\left.
\begin{array}{ll}
   \,\,\,\,\,\,\,\,\,\,u_{tt} =\alpha^2u_{xx},\,\,\,\,0<x<l,\,\,\,\,  t>-\infty; \\
   u(0,\,t)=\mu_{1}\,(t),\,\,\,t>-\infty; \\
   u(l,\,t)=\mu_{2}\,(l),
 \end{array}\right\}\tag{\(I_{0}\)}
\end{equation*}
которую будем называть задачей \((I_{0})\). Очевидно, что \(\bar{u}^{(1)} (x,\,t)\)\, \text{и}  \(\bar{u}^{(2)} (x,\,t)\)\, являются решениями задачи \((I_{0})\) при граничных условиях\begin{gather*}
    \bar{u}^{(1)}(0,\,t)=0,\quad \bar{u}^{(1)}(l,\,t)=A\,\cos\, \omega t\,, \\
    \bar{u}^{(2)}(0,\,t)=0,\quad \bar{u}^{(2)}(l,\,t)=A\,\sin\, \omega t\,.
\end{gather*}



Решение задачи при \(\alpha\) = 0 существует не всегда. Если частота вынужденных колебаний \(\omega\) совпадает с собственной частотой \(\omega_{n}\) колебаний струныс закрепленными концами\begin{gather*}
    \omega=\omega_{n}=\frac{\pi n}{l}a,
\end{gather*}
то знаменатель в формулах для \(\bar{u}^{(1)}\) и \(\bar{u}^{(2)}\)  обращается в нуль и решения задачи без начальных условий не существует.

Этот факт имеет простой физический смысл: при \( \omega=\omega_{n}\) наступает резонанс, т.е. не существует установившегося режима. Амплитуда, начиная с некоторого момента \( t=t_{0}\) неограниченно нарастает.

При наличии трения \( (a\neq0)\) установившийся режим возможен при любом \(\omega\) , так как \(\sin\,kl\neq0\) при комплексном \( k\),
\end{document}

